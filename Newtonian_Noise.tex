\section{Newtonian Noise}
\subsection{Current State of the Art}

\paragraph{Mitigation of atmospheric NN}
\begin{itemize}
\item Achieved below 2mHz in gravimeters using pressure sensors
\item NN spectra calculated based on highly-simplified models
\item Correlation studies between microphones carried out at Virgo
\end{itemize}

\paragraph{Mitigation of seismic NN}
\begin{itemize}
\item Achieved factor 1000 suppression of seismic signals in seismometers with underground array at Homestake
\item Achieved factor close to 100 suppression of seismic signals in seismometers with surface array at Hanford
\item Achieved factor >10 suppression of tilt signal in cBRS tiltmeter with surface array at Hanford
\item Analytical modeling of NN completed
\item Analytical modeling and numerical optimization of NN cancellation almost complete (missing the case of NN cancellation in underground arrays)
\end{itemize}

\paragraph{Site selection targets}
\begin{itemize}
\item Ambient seismic noise levels mostly understood from world-wide studies of impact of geology, distance to major cities, and coast
\item Some studies of geology made at existing detector sites and at Homestake
\item Seismic scattering from topography studied in linear order and only to obtain scattering coefficients
\end{itemize}

\paragraph{Infrastructure}
\begin{itemize}
\item Sound and seismic noise between about 5Hz and 50Hz are dominated by sources part of LIGO/Virgo infrastructure (e.g., ventilation) 
\item Stronger ground tilt is sometimes produced by wind pushing on buildings (problem clearly identified at LIGO sites); wind can also produce seismic noise at higher frequencies when interacting with rough surface topography
\end{itemize}

\subsection{Requirements}
\subsubsection{3G initial - surface}
\paragraph{Mitigation of atmospheric NN}
\begin{itemize}
\item Numerical simulations of atmospheric density perturbations
\item Analytical models of sound NN in terms of correlation functions (i.e., going beyond the plane-wave approximation)
\end{itemize}

\paragraph{Site selection targets}
\begin{itemize}
\item Investigate the impact of seismic scattering on NN cancellation
\end{itemize}

\paragraph{Infrastructure}
\begin{itemize}
\item Develop plans for low-noise detector infrastructure (with respect to sound and seismic disturbances)
\item Develop plans to avoid wind tilt at surface detectors
\end{itemize}

\subsubsection{3G initial - underground}
\paragraph{Mitigation of seismic NN}
\begin{itemize}
\item Analytic models of seismic NN from body waves in terms of correlation functions
\item Based on results from first item, calculate optimal underground arrays for cancellation
\end{itemize}

\paragraph{Mitigation of atmospheric NN}
\begin{itemize}
\item Investigate suppression of atmospheric NN as a function of detector depth using more realistic atmospheric models
\end{itemize}

\paragraph{Site selection}
\begin{itemize}
\item Investigate the impact of seismic scattering on NN cancellation
\end{itemize}

\paragraph{Infrastructure}
\begin{itemize}
\item Develop plans for low-noise detector infrastructure (with respect to sound and seismic disturbances)
\end{itemize}

\subsubsection{Future - surface}
\paragraph{Mitigation of atmospheric NN}
\begin{itemize}
\item Development of suitable instrumentation (e.g., LIDAR, effective wind shields for microphones) for NN cancellation
\end{itemize}

\paragraph{Site selection targets}
\begin{itemize}
\item Investigate the impact of seismic scattering on NN cancellation
\end{itemize}

\subsubsection{Future - underground}
\paragraph{Site selection targets}
\begin{itemize}
\item Investigate the impact of seismic scattering on NN cancellation
\end{itemize}

\subsection{Pathways and required facilities}
\subsection{Type of collaboration required:  small/large}
\subsection{Suggested mechanisms}
\subsection{Impact/relation to 2G and upgrades}