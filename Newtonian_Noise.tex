\section{Newtonian Noise}
Newtonian noise (NN) is predicted to be one of the limiting noise sources in third-generation detectors at frequencies below 30\,Hz \cite{Saulson:NN,Har2015}. It is produced by seismic fields, atmospheric sound and temperature fields, and vibrating infrastructure \cite{HuTh1998,BeEA1998,Cre2008,FiEA2018,Har2015}. Mitigation of NN can be achieved by suppressing density perturbations in the environment of the test masses \cite{HaHi2014}, and by cancellation of NN in GW data using environmental sensors \cite{Cel2000,CoEA2016a}.

In second-generation detectors, NN is mostly produced by detector infrastructure such as pumps and ventilation fans generating seismic and sound waves, but the natural environment will become more important in third-generation detectors, since, (a), we have learned and will still learn how to build detector infrastructure that does not significantly disturb the environment in the NN frequency band, and/or (b), our targets for NN cancellation might be so ambitious that we do not only care about the dominant anthropogenic perturbations, but also about weaker disturbances by nature. Therefore, NN research has a potentially big impact on site selection \cite{BeEA2010}. 

R\&D tasks can be categorized as follows:
\begin{itemize}
\item NN modeling
\item NN mitigation
\item Impact of NN on site selection and detector infrastructure
\end{itemize}
These items further divide according to the various sources of NN, e.g., seismic and atmospheric fields. Today, activities are ongoing in all three categories. Modeling of NN still requires efforts especially to take into account details of local geology for precise NN estimation, as well as for the prediction of atmospheric NN including turbulence and scattering of sound waves. 

Third-generation detectors might be constructed underground or at the surface. The modeling and cancellation of NN will be substantially different in these two cases, which means that R\&D tasks can often be clearly divided into surface and underground. 

\subsection{Current State of the Art}
\subsubsection*{Modeling of seismic NN}
\noindent Analytic expressions exist to calculate seismic NN$\ldots$
\begin{itemize}
\item in homogeneous half spaces with cavities for arbitrary wave polarizations \cite{Har2015}.
\item in laterally homogeneous half spaces for a variety of wave polarizations (shear, compressional, Rayleigh, Stoneley, and a few others) \cite{Har2015}.
\item from seismic body waves scattered from spherical cavities \cite{Har2015}.
\item from specific seismic sources such as point forces and point moments in homogeneous half spaces \cite{HaEA2015,Har2016}. 
\end{itemize}
Numerical simulations were performed to $\ldots$
\begin{itemize}
\item simulate seismic fields from point forces in laterally homogeneous half spaces \cite{BeEA2010c}. 
\item compare results from different numerical simulations also with analytic solutions (Lamb's problem).
\item simulate seismic fields in media with arbitrary surface profile.
\item simulate NN including self-gravity effects (i.e., gravity induced ground motion) \cite{VaEA2017}.
\end{itemize}

\subsubsection*{Modeling of atmospheric NN}
\noindent Analytic expressions exist to calculate atmospheric NN$\ldots$
\begin{itemize}
\item from homogeneous temperature fields in laminar flows \cite{Cre2008}.
\item from turbulence induced pressure fluctuations (Lighthill process) \cite{CaAl2009,Har2015}.
\item from homogeneous sound fields \cite{Cre2008,FiEA2018}.
\item from atmospheric shock waves \cite{Cre2008,Har2015}.
\end{itemize}
Numerical simulations were performed to $\ldots$
\begin{itemize}
\item take into account details of the atmosphere such as division into indoor and outdoor atmosphere and exclusion volume from vacuum tanks \cite{FiEA2018}.
\item to estimate sound and temperature NN underground \cite{FiEA2018}.
\end{itemize}

\subsubsection*{Mitigation of seismic NN}
\begin{itemize}
\item Surface seismometer arrays optimized for NN cancellation were calculated using analytic and numerical methods \cite{Har2015,CoEA2016a}. 
\item Analytic expressions were derived for Rayleigh waves to express NN and correlations between NN and seismometers in terms of observed seismic correlations \cite{Har2015,CoEA2016a}.
\item The potential impact of surface topography on NN cancellation was investigated \cite{CoHa2012}.
\item Tiltmeter signal suppressed by more than an order of magnitude using Wiener-filtered data from seismometer array, which serves as proxy of NN cancellation \cite{HaVe2016,CoEA2018}.
\item Factor 1000 (100) suppression of seismic signals in seismometers using Wiener-filtered data from underground array at Homestake (surface array at LIGO Hanford) \cite{CoEA2014,CoEA2018}.
\item Extensive seismic array measurements were performed at Sanford Underground Research Facility, LIGO Hanford, and Virgo.
\end{itemize}

\subsubsection*{Mitigation of atmospheric NN}
\begin{itemize}
\item Achieved below 2\,mHz in gravimeters using pressure sensors \cite{HCW2007}.
\item Extensive sound correlation measurements were performed at Virgo.
\end{itemize}

\subsubsection*{Impact of NN on site selection and detector infrastructure}
\begin{itemize}
\item Ambient seismic noise levels mostly understood from world-wide studies of impact of geology, distance to major cities, and coast
\item Some studies of geology made at existing detector sites and at Homestake
\item Seismic scattering from topography studied in linear order and only to obtain scattering coefficients
\item Sound and seismic noise between about 5Hz and 50Hz are dominated by sources that are part of LIGO/Virgo infrastructure (e.g., ventilation)
\item Concrete plans are being developed for infrastructure changes at Virgo to mitigate NN from sound fields 
\end{itemize}

The following groups are currently pursuing NN activities for second and third-generation detectors (only including groups who communicated activities in the past few years):
\begin{itemize}
\item APC: modeling and cancellation of atmospheric NN, characterization of sound fields
\item Caltech: sensor array optimization, development of NN cancellation filters
\item EGO: characterization of seismic and sound fields, infrastructure modifications
\item GSSI: atmospheric and seismic NN modeling and cancellation, sensor array optimization, characterization of seismic fields
\item INFN-Genova: numerical simulations of seismic fields
\item INFN-Naples: development of tiltmeter for NN cancellation 
\item INFN-Pisa: NN modeling and analytic investigations of NN cancellation
\item INFN-Roma 3 (numerical simulations of seismic fields
\item Nikhef: numerical simulations and measurements of seismic fields
\item Korean Gravitational-Wave Group: seismic NN cancellation
\item LIGO Hanford: characterization of seismic fields, noise cancellation
\item Polgrav: measurements of seismic fields and development of NN cancellation system
\item University of Minnesota: characterization and modeling of underground seismic fields
\item University of Washington: development of tiltmeter for NN cancellation
\end{itemize}


\subsection{Requirements}
\subsubsection{3G initial - surface}
\paragraph{Mitigation of atmospheric NN}
\begin{itemize}
\item Numerical simulations of atmospheric density perturbations
\item Analytical models of sound NN in terms of correlation functions (i.e., going beyond the plane-wave approximation)
\end{itemize}

\paragraph{Site selection targets}
\begin{itemize}
\item Investigate the impact of seismic scattering on NN cancellation
\end{itemize}

\paragraph{Infrastructure}
\begin{itemize}
\item Develop plans for low-noise detector infrastructure (with respect to sound and seismic disturbances)
\item Develop plans to avoid wind tilt at surface detectors
\end{itemize}

\subsubsection{3G initial - underground}
\paragraph{Mitigation of seismic NN}
\begin{itemize}
\item Analytic models of seismic NN from body waves in terms of correlation functions
\item Based on results from first item, calculate optimal underground arrays for cancellation
\end{itemize}

\paragraph{Mitigation of atmospheric NN}
\begin{itemize}
\item Investigate suppression of atmospheric NN as a function of detector depth using more realistic atmospheric models
\end{itemize}

\paragraph{Site selection}
\begin{itemize}
\item Investigate the impact of seismic scattering on NN cancellation
\end{itemize}

\paragraph{Infrastructure}
\begin{itemize}
\item Develop plans for low-noise detector infrastructure (with respect to sound and seismic disturbances)
\end{itemize}

\subsubsection{Future - surface}
\paragraph{Mitigation of atmospheric NN}
\begin{itemize}
\item Development of suitable instrumentation (e.g., LIDAR, effective wind shields for microphones) for NN cancellation
\end{itemize}

\paragraph{Site selection targets}
\begin{itemize}
\item Investigate the impact of seismic scattering on NN cancellation
\end{itemize}

\subsubsection{Future - underground}
\paragraph{Site selection targets}
\begin{itemize}
\item Investigate the impact of seismic scattering on NN cancellation
\end{itemize}

\subsection{Pathways and required facilities}
I suggest to remove this section from the low-frequency part. Only far-future projects might need specific facilities (e.g., to test LIDAR systems). Also the pathway is not that interesting to describe since the optimal strategy is simply to do everything necessary at once... There is very little dependence between tasks.

\subsection{Type of collaboration required:  small/large}
\begin{itemize}
\item Collaboration is strongly recommended with respect to site characterization. It can take years to understand how to set up high-quality environmental measurements, what is relevant to measure, and how to analyze environmental data.
\item Collaboration is also encouraged for numerical simulations to share generic knowledge, e.g., of how to assess accuracy of numerical simulations (e.g., seismic and atmospheric). 
\end{itemize}

\subsection{Suggested mechanisms}

\subsection{Impact/relation to 2G and upgrades}
\begin{itemize}
\item NN cancellation techniques will evolve continuously from 2G to 3G surface detectors. 
\item Solving the problem of NN cancellation in underground 3G detectors is only weakly related to NN cancellation in surface 2G/+ detectors. 
\item Certainly, since we have never actually canceled NN, first experience with this technique in 2G detectors might bring valuable insight, with potentially large impact on 3G NN mitigation strategies.
\end{itemize}
