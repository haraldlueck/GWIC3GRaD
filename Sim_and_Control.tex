\section{Simulation and Controls}
\label{sec:SimControls}
\redtext{Freise, Adhikari}
Wiki: \url{https://gilsay.physics.gla.ac.uk/dokuwiki/doku.php?id=gwic_3g_rad:simulations_and_controls}

List of r+d topics collected on and tranferred from the Wiki:
\begin{itemize}
\item advancing IFO control systems:  alignment sensing and control,
  long baseline / high Finesse length sensing and control,  parametric
  instabilities (opto-mechanical, as well active/passive damping),
  mode sensing, wavefront sensing and control, compatibility of
  optical systems with LG modes.
\item modern controls: optimal MIMO (multiple-input multiple-output)
  sensing and control, noise-subtraction (online, offline, supervised
  learning),  robust, automatic filter design, sensor blending, Neural
  Networks for feedback optimization
\item seismic isolation: global low frequency control, tilt sensing
  (6D sensing),  low-noise local controls
\item quantum noise reduction: filter cavity control (passive,
  active),  optical spring control (passive, active), fundamental
  quantum limit, active filters, polarisation schemes, loss, cross
  coupling, speedmeter schemes, optical properties, noise coupling,
  control high-frequency detectors (generally, not-full-band
  detectors), integration of Quantum Information error correction
  techniques
\item suspension Design: Finite Element Analysis of mechanical
  structures (thermal noise and vibration isolation), Electro-magnetic
  suspensions, High Q Blade Springs (sapphire, silicon, silica, glassy
  metals, etc.), Force noise from electric charge and electric field fluctuations
\item scattering: simulations of scattered light, simulations on
  active mitigation of backscatter
\end{itemize}


\subsection{Simulations}
Interferometric gravitational-wave detectors are sufficiently
complicated optical systems that detailed modeling is required for
design and performance studies.  The detector behaviour cannot be
modelled with commercially available optical simulations because
our requirements are very different from conventional systems. 
The first simulation tools for modelling laser interferometers
for gravitational wave detectors were developed to aid the design
and analysis of prototype interferometers and the first generation
of detectors, leading to two large time-domain packkages, 
 the End-to-End model for LIGO also known as `E2E'  and `SIESTA'
in the Virgo collaboration, plus the first generations of FFT codes.
For the design of the next generation, known collectively as
`advanced detectors', the need for better simulation tools was
highlighted. A dedicated effort spun out from the 
LSC Advanced Interferometer Configuration
working group resulted in significantly advanced smaller
tools, for example for FFT-based modelling of the thermal deformation 
of the main optics,  or fast mode-based tools for control design, 
noise propagation and commissioning. 

The planned upgrades to the detectors include either new technologies
or envisage pushing detector parameters closer to their limits. In both cases the 
detector design and commissioning face new challenges, as
previously used assumptions and simplifications can 
no longer be used. Simulation tools and the understanding
of interferometer modelling have to be advanced ahead of time
in order to be able to provide the essential design and commissioning
support. For example, some simulation
tools are being extended to allow the study of advanced 
quantum noise reduction schemes, more accurate predications
of scattering and other opo-mechanical couplings, such
as parametric instabilities. Further, the integration of mechanics and
electronic into mode based tools has been started to aid the 
controls effort, e.g. for modern controls with global multi-input 
multi-output strategies.

\subsection{Current State of the Art}
\begin{itemize}
\item Simulations which use the FFT for propagating paraxial beams such as
SIS or OSCAR can be used to make
detailed predictions about the impact of optical phase errors (surface
roughness, phase distortions, etc.)  and finite aperture sizes.
\redtext{H. Yamamoto, Degallaix}
\item For understanding interferometer dynamics over a wide range of
conditions which do not allow for linearization time-domain
simulations are the most 
appropriate tools. However, this type of simulation is slow and
is in practise used only for modelling the lowest transverse-spatial
modes.
\redtext{ ? }
\item 
The commissioning of advanced detectors, as well as the design of further 
upgrades have shown that the combination of frequency domain 
simulations with paraxial beam propagation has become crucial
to understand interferometer limitations. Geometric instabilities and thermal
beam distortions, both due to higher laser power, can dominate the
interferometer behaviour. Modelling tasks related
to alignment sensing and control, parametric instabilities, or simply
the  effect of mode mismatch or beam clipping on the control systems and
ultimately on the gravitational wave signal have taken center stage.
Simulation tools that use higher-order mode
expansion to describe the laser beam properties can combine the speed
of frequency domain tools with the power of paraxial
beam propagation.  Further development is ongoing to improve their performance and to
add more advanced features.
\redtext{ Freise, Brown}
\item In addition to numerical interferometer models, the community
  has developed configuration level simulation that operate at a
  higher level in that the response of a given optical configuration
  is symbolically computed and parameterized.
These symbolic computations can be
compared with the more detailed numerical results given by frequency
domain codes.  While inappropriate for detailed simulation of the
optical plant, this approach is very effective for an initial
exploration of a parameter space for a variety of optical configurations.  \redtext{ ? }
\end{itemize}

\subsection{Requirements}
Description of simulation requirements that follow from example 
challenges in detector design and commissioning.
\begin{enumerate}
\item Steady-state simulation (e.g. Optickle, Finesse)
\item time domain simulations (lock acquisition)
\item FFT models for spatial modeling
\item Parametric Instability (\redtext{S. Gras, C. Blair})
\item Kalman filters for thermal model of mirrors
\end{enumerate}

\subsubsection{3G initial}
Each improvement in detector sensitivity will uncover new unwanted 
noise coupling mechanisms that typically require post-hoc
mitigation strategies. This implies that for the designs of
future detectors, we must model and study potential designs at 
greater depths than previous detectors.
Initial interferometers in new faculties
could be based on the technologies developed for the
upgrades of current facilities, such as LIGO Voyager.
The envisaged upgrades will possibly introduce 
significant changes, such as cryogenic
suspensions and a different laser wavelength. 
The main challenge will be the operation  at very high circulating 
power (several MW). For the initial design a more  consistent
modelling of control noises is required  to reduce the risk of
low-frequency excess noise already at the design stage.

\begin{itemize}
\item High-power operation at low optical loss. Small asymmetries
in absorption can increase the effective optical loss for squeezed
states. Modelling of squeezed light in higher-order modes, improved
thermal compensation systems, improved arm and mode matching
techniques.
\item Parametric instabilities control, modelling of advanced mode
  dampers, development of a control scheme that
  allows tracking and selective damping of a large number of modes,
  requires more detailed modelling to predict individual modes.
\item Scattered light control, modelling of backscatter of detection
optics and internal interferometer scatter. Include injection of
noise with specific coherence into interferometer modelling tools.
\item Control design: better models of control schemes can be
  achieved by developing more effective tools
for the analysis of in-loop cross coupling of a mixed mechanical, 
optical and electronic system. It is expected that modern control
schemes as mention in  will be used in a subset of
systems at this time. 
\end{itemize}

\subsubsection{Future}
Over the lifetime of the new facilities being constructed in the
near future, we envisage a series of detector upgrades
to fully explore the facility limits.
Quantum noise reduction presents a challenge that
will most likely have the strongest impact on the interferometer
design. Newtonian noise reduction will be required.
Other priority modelling task for this stage include:

\begin{itemize}
\item Advanced quantum noise schemes, development of a 
  robust 'fundamental' quantum limit. Modelling of quantum
  correlations through complex MIMO (multiple in, multiple out)
  systems.
\item Modelling of non-linear optical elements, such as crystals
  (squeezing), active opto-mechanical elements (unstable filters)
  etc.  Development and implementation of realistic linearised
  couplings for these elements into optical models. 
\item Modelling of long-baseline and high-cavity finesse operation, 
including geometric consideration, control schemes and thermal
effects.
\item Study of optical configurations which rely strongly on polarisation 
schemes, requires the addition of light polarisation to interferometer models.
\item Newtonian noise reduction, advanced modelling of local sensing
  (6D) and global control strategies, a more advanced implementation 
of mechanical systems and seismic and Newtonian noise coupling in
interferometer models. Simulation of Newtonian noise based on ground 
noise measurements.
\item Investigation of alternative beam shapes for thermal noise
  reduction, modelling of auxiliary optical systems to study
  feasibility 
\end{itemize}


\subsection{Pathways and Required Facilities}
\begin{itemize}
\item Working groups
\item workshops
\item documentation, code maintenance 
\item Prototypes ?
\end{itemize}

\subsection{Type of Collaboration Required:  small/large}
\redtext{Freise}

\subsection{Suggested mechanisms}
\redtext{Freise}

\subsection{Impact/relation to 2G and upgrades}
\redtext{Freise}

% ----------------------------------------------------------------
\subsection{Controls}
\redtext{Adhikari}
\subsubsection{Current State of the Art}
\redtext{Ward, Coyne, Adhikari}

In most cases, classical control methods (essentially pole placement)
are used to minimize the spectral density of the particular 
error signal.

There are a few examples of the implementation of more modern techniques:
\begin{enumerate}
\item Global feed-forward of seismic noise to platforms
      and suspensions.

\item $\mu$-Synthesis approach for limited angular control

\item Feed-forward sensor noise subtraction 
      (removal of seismic noise from WFS)

\item Parametric Instability damping by PLL and damping through 
      aliasing of high frequency signals.
\end{enumerate}

\subsection{Examples}

\subsection{Alignment Sensing \& Controls}
\redtext{Grote}
The alignment fluctuations at low frequencies are difficult to control due to the large micro-seismic motion and the Sigg-Sidles instabilities. The feedback controls noise is a limiting noise source at the lowest end of the GW band.

\begin{itemize}
\item Better sensors
\item Better loops
\item Mirror spot image processing for centering
\item scattered light mitigation
\item vibration noise coupling to sensors
\end{itemize}


\subsection{A Quantum Interferometer Component: The Pre-Amplifier}
\redtext{D. Martynov, H. Miao}


\subsection{FEM of Mechanical Structures}
\redtext{G. Venugopalan, A. Markowitz}


