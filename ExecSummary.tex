\section{GWIC 3G R\&D Executive Summary}
\label{sec:Exec}

\redtext{(McClelland, Lueck)}

The next generation of gravitational-wave detector will require a tight coordination of R\&D topics.

Within this document, the major topics are described as well as the mechanisms for coordination.

The main conclusions are that:
\begin{enumerate}
\item A series of next-gen focused workshops will be held with clearly stated goals and workshop reports to be written and distributed to the GW community.

\item Each of the major R\&D tasks contains a list of stated goals with quantitative metrics as well as timelines.

\item This R\&D program to be updated bi-annually with a snapshot of the current work as well as the expected progress over the upcoming two decades.
\end{enumerate}

\textbf{Charge:} Coordination of the Ground-based GW Community R\&D:~develop and facilitate coordination mechanisms among the current and future planned and anticipated ground-based GW projects, including identification of common technologies and R\&D activities as well as comparison of the specific technical approaches to 3G detectors.

Including identifying primary (enabling or fundamental) and secondary (or technical) technologies \\


\noindent 

\noindent Enabling technologies are the main pillars on which the design is based. For ET these are cryogenics, silicon mirrors, squeezing, underground facilities. All these technologies enable ET to arrive approx to a factor 10 improvement w.r.t the design sensitivity of the advanced detectors (advanced LIGO and advanced Virgo). Shorter suspensions, cheaper vacuum pipes are not an enabling technology, because you can "solve" "just" paying more.\\

In the absence of any disruptive technique on the horizon we assume that the basic method of detecting GWs will remain suspended mass laser interferometry.  We divide the noise sources into fundamental and technical.    Fundamental noise sources arise from quantum  noise and Brownian noise.  Technical {\dots}

Impact on either displacement or sensing noise.  Any displacement noise directly affecting test mass motion can be reduced by increase interferometer length. Sensing noise  - how well can resolve the fringe; does not directly depend on length in general length independent to first order.

In the absence of a science case how do we set the goals?

Rule of thumb:  

\begin{enumerate}
\item  \textit{ minimum} factor of 10 reduction on advanced detector design dL (or h?) at each frequency;  

\item  conceivable factor given extrapolation of current state of the art.
\end{enumerate}

Like to know now:  desired low and high-frequency cutoffs

\begin{enumerate}
\item  The fundamental limiting noise sources are quantum noise, coating thermal noise, suspension thermal noise, and Newtonian noise,
\begin{itemize}
\item Reducing quantum noise requires high power lasers and highly squeezed quantum, low absorption (function of T?) and possibly speed meter configurations, high QE; heavy test masses.
\item  Reducing thermal noise:  reduce loss angle; changing beam parameters; reduce T;  low absorption test masses at longer wavelengths
\item NN:  cancellation; environment.
\item  Many other technical issues to be dealt with including scatter, mode-matching, instabilities; alignment.
\item length scaling.
\end{itemize}
\end{enumerate}

 R\&D required:  lasers, cryogenics, coatings; materials, \\
 
\paragraph{Recommendations:}

\begin{itemize}

\item \noindent \textbf{Lasers and squeezers:}  wavelength TBD; requirements TBD; but techniques well established. 
 Well covered within the community.\\    Cost:  low 

\textbf{R1:}  on track; globally organized collaboration not essential

\item \noindent \textbf{Photodetectors:}  no issue up to to 1.6 microns.  Major effort required to produce high QE at 2 microns.  Requires `industrial' partnership. \\ Cost:  medium

\textbf{R2:}  form international photodetector consortium with funding contributions for industry cycles

\item \noindent \textbf{Test masses (substrates): } no issue at 1 micron fused silica(?).  Longer wavelength requires materials such as silicon, sapphire, other;  issues of growth, purity;  Requires industrial partnerships; \\ 
Cost:  high

\textbf{R3:}  form international substrate consortium with funding for industry cycles.

\item \noindent \textbf{Coatings: }major issue regardless of wavelength.  Two main and most promising research areas (i) Amorphous; (ii) crystalline.  Other options include coating free techniques.  Needs interaction with `industry' (large coaters) partners; many cycles; extensive testing = expensive .\\   
Cost:  high

\textbf{R4:}  form international coating consortium with funding for industry cycles of a size and scale for test, measurement and analysis.

\item \noindent \textbf{Cryogenics:}  120K; 20K; 4K.  Lowers all thermal noise; exploitation of reduce thermal coefficient of expansion could be crucial for high power operation.  Issue: Heat removal without noise coupling; strongly coupled to coatings and Test masses.\\     
Cost: high

\textbf{R5:} Support for a prototype cooled silicon phase noise interferometer test bed. Support for collaboration with KAGRA gaining hands-on experience with cryogenic interferometer. Global?


\item \noindent \textbf{NN cancellation: } in- principle idea; yet to be demonstrated in practice. Grow research community; measure and cancel NN on prototypes or on 2G+; extensive modelling. Low cost? Requirements differ between 3G underground and above-ground. \\     
Cost:  medium

\textbf{R6:}  form international NN consortium with funding contributions for  demonstrators.

\end{itemize}


\noindent \textbf{\textit{Overall Recommendation:  Form international consortia to work on key problems with industry partners; establish leadership and organization structure with teeth (ie controls purse strings); seek funding via common proposal submitted across funding agencies.}}

\noindent \textbf{\textit{}}

\begin{enumerate}
\item \textbf{\textit{ }} Secondary (technical) noise
\end{enumerate}

\noindent Includes control systems; control of instabilities; alignment; modematching

\noindent 

\noindent 

\noindent 

\noindent 

\begin{enumerate}
\item   \textbf{Facilities}: 
\end{enumerate}