\section{Quantum Enhancements}
\subsection{Quantum Noise Reduction Techniques under consideration}
Quick overview of the most prominent quantum noise reduction techniques currently persued. Maybe this could be done in a separate explanation box rather 
than in the main text of the document?
\subsection{Current State of the Art}
\begin{itemize}
\item Dual-recycled, Fabry-Perot Michelson
\item Injection of squeezed light, obtaining about 4dB max
\item A+ and Advanced Virgo+ will test frequency dependent squeezing with short (300\,m) filtercavities, targetting 6dB effective squeezing.
\item Balanced Homodyne readout will be part of A+, allowing in principle to change readout quadrature.
\item Quick list of current losses for 1064nm: QE PDs, faraday isolators etc.
\end{itemize}
\subsection{Requirements}
Need to start here with a quick discussion on the relation of quantum noise 
and laser wavelength: To first order quantum noise reduction schemes are independent of wavelength so all the drivers for changing towards longer wavelength come from other noise sources.Hence in the following assume that each of the discussed quantum noise reduction techniques can be realised equally well for any laser wavelength. However, we need to keep in mind that there could 
be potential somestoppers, e.g. if it would turn out that there are no high 
quantum efficiency photo diodes available at 2 micron, which could lead for quantum noise considerations to rule out certain wavelengths.  
\subsubsection{3G initial}
\begin{itemize}
\item \textbf{Result driven: 10\,dB effective quantum noise reduction over 2G for frequencies at mid and high frequency, plus pushing the radiation pressure noise below a few Hertz.}
\item Current baselines: ET and CE designs are currently based on tuned (ET-HF and CE) or detuned (ET-LF) dual-recycled Fabry Perot Michelson interferometers, with frequency dependent squeezing created with km-scale filter cavities, combined with heavy test masses of the order 200\,kg to reduce radiation pressure noise at low frequencies. 
\item Alternative approaches: Several alternative approaches are currently researched which provide more cost effective ways to improve the same senstivity as the baselines mentioned above. Examples include conditional squeezing which uses an EPR measurement for generation of frequency dependent squeezed light and therefore removes the need for long baseline filtercavities; speedmeter interferometers which inherently surpress back action noise.     
\end{itemize}
\subsubsection{future}
No limit. The more reduction the merrier. (Obviously need some better words for this section. Might need to discuss within full committee what level of speculation/optyimism one should put in here to make it consident with the other sections?)
\subsection{Pathways and required facilities}
\begin{itemize}
\item Bench mark parameter set, for evaluation of different quantum noise reduction schemes. 
\item Deticated prototype test programmes for full interferometer schemes and low noise tests.
\item Table top tests of interferometer building blocks, such as low loss optics, high QE photo diodes. 
\end{itemize}
\subsection{Type of collaboration required:  small/large}
\begin{itemize}
\item From feedback we received from community, the key that is needed is not 
necessarily more collaboration (though this si also welcome), but rather more coordination, to make sure there are prototype experiments testing all the different promising configurations and we do not miss to develop one. 
\end{itemize}
\subsection{Suggested mechanisms}
\begin{itemize}
\item Community feedback asked for annual quantum noise workshop with special focus to bring together theorists and experimentalists. 
\item Prototype coordination (partly happening already via LSC MOUs). Which body would take this on to cover more than just LSC?
\end{itemize}
\subsection{Impact/relation to 2G and upgrades}
\begin{itemize}
\item 2G consolidates state of the art and test baseline scheme, i.e. frequency
dependent squeezing. 
\item Benefit for 3G programme for 2G: Potentially improved sensivity and faster commissioning plus risk reduction.
\end{itemize}