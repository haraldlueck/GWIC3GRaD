\section{Introduction}

\textbf{Charge of the GWIC subcommittee:}\\
Coordination of the Ground-based GW Community R\&D: develop and facilitate coordination mechanisms among the current and future planned and anticipated ground-based GW projects, including identification of common technologies and R\&D activities as well as comparison of the specific technical approaches to 3G detectors.\\
Possible support for coordination of 2G observing and 3G construction schedules $\rightarrow$ Governance?\\
Including identifying primary (enabling or fundamental) and secondary (or technical) technologies
\par
\vskip\baselineskip
Current (2018) gravitational wave detectors have not yet reached their design sensitivity. At the time of report writing there is a gap of a factor of about 3 between the current sensitivity and design. Plans for pushing the sensitivity beyond the advanced detector design for advanced LIGO and advanced Virgo in the existing infrastructure exist at various levels of maturity and are shortly outlined in this report. \\
Conceptual designs and ideas for the third generation of GWD exist for the Einstein Telescope on the European side and for Cosmic Explorer on the US side. (short description and pointers to documents...\href{http://www.et-gw.eu/index.php/etdsdocument}{ET Design study document} )

Although the main focus of this GWIC 3G R\&D subcommittee is the 3rd generation of gravitational wave detectors there is a large technological overlap between the extensions of the advanced detectors and 3G. Hence foreseen and ongoing R\&D efforts for the ``advanced +'' generation will also be included in this report.
The aim of this report is to provide an overview of current research activities and research locations, to identify gaps, to estimate time scales, (maybe financial and human resources requirements) and to provide assistance in the selection of future techniques. This report should become  a living document that is constantly updated and adapted to current developments. Time scales and levels of maturity for the different topics can vary greatly. While research and decisions on third-generation infrastructure have to be pushed forward quickly, such that (from the technological readiness point of view) construction could begin in the middle of the next decade, decisions on many details, e.g. of read-out concepts and controls aspects, are less urgent.
\par
\vskip\baselineskip
Subdivided into topics this report describes
\begin{itemize}
\item{Current State of the Art}
\item{Requirements}
\item{3G initial}
\item{future}
\item{Pathways and required facilities}
\item{Type of collaboration required:  small/large}
\item{Suggested mechanisms}
\item{Impact/relation to 2G and upgrades}
\end{itemize}

The individual sections will point out the potential for improving sensitivity in different frequency ranges. The classification of the relevance of the improvement in sensitivity must be made in connection with the report of the Science Case Committee. 
\par
The updating of the document needs to be discussed. Does an annual renewal make sense? How can this be included in the work on the White papers? How is the coordination with the subsystem working groups in LIGO and Virgo coordinated? The basic assumptions for this report are that the sensitivity of the current advanced detectors will be increased to the limits of their infrastructure and that new infrastructures for the third generation will be built from the mid-2020s onwards. It is assumed that the Einstein telescope will be realized on the European side and a new large detector Cosmic Explorer on the US side. Like the two advanced LIGO detectors, advanced Virgo and LIGO India, the KAGRA detector will be improved in its sensitivity to its technical limits. All detectors are assumed to have a remaining service life of at least 15 to 20 years. The research work required for the third generation will be carried out in parallel with the upgrades of the current detectors, so that increased personnel and financial requirements must be assumed. At the present time, there is a strong increase in interest from scientists outside the gravitational wave community, so that the difficulty in recruiting a sufficient number of knowledgeable people is currently not expected to become a problem.

All current gravitational wave detectors are based on the Michelson interferometer principle with arm cavities and dual recycling.
There is still a gap of a factor 3 between the current detector sensitivity and the plans for the advanced generation. In addition to the new technologies included in the advanced generation design, Squeezing is already being implemented in the current detectors and can (when used together with full laser power) yield an improvement in the high-frequency range of up to a factor of 2. 
The technology for the ``voyager class'' of detectors is still immature, but may be of importance for the third generation.

