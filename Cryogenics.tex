\section{Cryogenics}
\label{s:Cryogenics}
A feature of most designs for future interferometers, is the 
cryogenic~\footnote{which we define here to be less than 123\,K; the range from 80\,--\,123\,K may be considered as "high temperature cryogenics"} operation of the test mass mirrors\cite{ET, Voyager}.

In the GW field, there is a long history of using cryogenic resonant mass detectors\cite{coldBars}, as well as the recent operation of the CLIO and KAGRA interferometers in Japan.

\paragraph{Thermal Noise}\mbox{}\\
The primary benefit of low temperature operation of the mirrors is through the reduced thermal noise:
\begin{itemize}
\item Brownian noise of mirror substrate
\item Brownian noise of mirror coating
\item thermo-optic (thermo-elastic and thermo-refractive) noise
\end{itemize}
The relation between the dissipation and the power spectrum of the noise is described by Callen's Fluctuation-Dissipation Theorem~\cite{CaWe1951, Kubo:FDT, Callen:1959}:
\begin{equation}
S_x(f) = \frac{k_B T}{\pi^2 f^2} \left| Re \big[ Y(f) \big]\right|
\label{eq:FDT}
\end{equation}
and so we see that the displacement noise, $x(f)$, scales as $\sqrt{T}$. More significantly, many of the material properties of mirror substrates (e.g. sapphire and silicon) and coatings (e.g. GaAs, GaP, $\alpha$-Si), scale in a favorable way with decreasing temperature. This makes the improvement in the noise substantially larger than the naive $\sqrt{T}$ scaling.

\paragraph{Robust Operation}\mbox{}\\
In addition to the noise improvement, there are a number of operational issues affected by a low temperature environment:
\begin{enumerate}
\item Increased thermal conductivity in crystalline substrates; this dramatically reduced the thermal gradients in the mirror, and thereby, the induced wavefront distortions due to thermo-elastic deformations of the mirror surface and thermo-refractive lensing in the substrate bulk.

\item zero thermal expansion in silicon at 18 and 123\,K

\item ???
\end{enumerate}



\subsection{Current State of the Art}
\begin{itemize}
\item cryogenic bars
\item CLIO
\item KAGRA
\item silicon Fabry-Perot cavities for atomic clocks (JILA, PTB)
\end{itemize}
\subsection{Requirements}
\subsubsection{3G initial}
\subsubsection{future}
\subsection{Pathways and required facilities}
What is needed to make a useful prototype?

What requirements on these prototypes?
\subsection{Type of collaboration required:  small/large}
\begin{enumerate}
\item 
\end{enumerate}
\subsection{Suggested mechanisms}

\subsection{Impact/relation to 2G and upgrades}


\section{Cryogenics below 20 K}
The technological effort to cool down the mirrors at temperatures around 4 K   is not  different from that used to  operate the interferometer at 20K.
The engineering plant is  almost identical both if we use cryogenic fluids or commercial cryocoolers. The main difference in the improvement of the heat extraction from the payload and the reduction of the thermal inputs.
It follows that the main issue is the optimisation of the thermal links  used to transmit the refrigeration power and the fibers for the heat extraction from the mirror.
The main outcome to be below 20 K is to take advantage of the  superconductivity  of materials used with the possibility to develop  very low noise  actuators. 
This can have a large impact on the reduction of the control noise, a crucial issue, which affect the sensitivity  in particular in the low frequency range of the detector bandwidth.
 
The  control of test masses plays a crucial role in gravitational wave detectors with independent test masses. Payload local control system is meant to slow-down and align the test masses as interferometer’s mirrors, driving their dynamics within the gain range of automatic error signals (wavefront sensing for angles and locking for longitudinal position). In order to preserve the quasi-inertial state guaranteed at the level of payload suspension point by the seismic attenuator system, only internal forces should be used. The 3G digital/analogic control system needs to be improved to that of the advanced detectors, namely by reducing by more than one order of magnitude the overall low frequency noise re-injection during locking force reallocation to payload stages. The related R\&D will naturally follow the developments of advanced detector implementations. The task will be complicated by the presence of cryostats surrounding the vacuum chambers hosting the payloads and the use of extra viewports  should be avoided to limit the thermal input. 
As consequence the sensing access to the mirror position without significant perturbation of cryostat performance in mirror cooling. This can be done by an extensive use of optical fiber sensors or even inventing new capacitive  detectors with  SQUID preamplifiers from which an almost noiseless error signal can be extracted to drive the actuators. This last element is crucial for applying forces on the different  stages of the payload. The actuators acting on the marionette and the reaction mass are designed to allow active control of the locking and alignment during operation.
Small and fast corrections of the mirror position can be obtained, through the marionette, if the mechanical transfer function of the system is taken into account. For example, due to the response of pendulum mechanical filters, the displacement amplitude of the mirror face with respect to that applied to the marionette arms decreases with the frequency.
In the advanced detector configuration  magnets are placed on the marionette arms and coils  are set in front of each of them. Through these actuators, it is possible to steer, around the directions perpendicular  to the line  joining the magnet pairs. In order to design the coil - magnet actuators for the marionette and the mirror of the LF interferometer, several constraints must be taken into account.
We should consider the effect of the magnetic noise produced by the magnet at 4.2 K,  its magnetisation change due to the cooling, the current noise of the coil and the power dissipated by the current flowing in the coil. The use of superconducting wires for the coils is the obvious solution to kill the last contribution, while for the magnet itself it has been demonstrated that with a suitable material choice the Barkausen noise can be kept  well below the threshold set by the low frequency sensitivity of a 3G detector {\color{red} P. Falferi, Classical and Quantum Gravity, IOP Publishing, 2011, 28 (14), pp.145005}.

In this temperature range it is possible also  to develop new push-pull actuators based on Meissner effect, whit the great advantage to replace the magnets glued on the mirror or attached to he marionette arms with superconducting thin films coated on the surfaces. 


\section{R\&D for lowering the temperature}
The optimisation of the mirror suspension fibers is the  main action to be pursued in order to keep the mirror at temperature below 20 K.
This implies to explore new geometric configurations  and new materials being constrained by the need  to have a low suspension thermal noise and at the same time an efficient path to extract the heat power from the mirror. 

In principle a 4 K cryostat is not  different from that for 20K.  In fact, the final temperature of a double stage cryocooler is in the 4 K  range, so   this device  can be stil used to  achieve this temperature.  It is far obvious that, to get a lower temperature we should reduce the thermal input and/or  use more refrigeration power. An increased number of cry-cooler can imply higher vibration.  In this respect a R\&D   devoted to design and construct  silent refrigerator machine is highly desirable. 
Several ideas have been proposed   example the use of PT cryocooler, with symmetric cold heads and   driven  at the same helium wave frequencyin  opposite phase. However, these devices are not commercial yet and R\&D carried on i collaboration with a cry-oindustry is highly recommended..
 In order to increase the cooling efficiency and to get a further reduction of the thermal noise contribution, the most powerful approach is to make use of the liquid helium II (He II), which has great advantages. It limits the vibration noise associated to the other cryogenic fluids  and it provides a powerful way for extracting the heat from the mirrors. Modern large engineering projects for high-energy physics require thermostatic control of working components at the level of 1.8 - 2 K and are constructed with lengths of channels containing He II. The uniqueness of He II is that it contains a superfluid component with zero entropy, which moves through other liquids and solids with zero friction to an extent dependent on the temperature of the liquid. He II is a liquid of extremely low viscosity and very high heat capacity, which prevents small transient temperature fluctuations. Moreover, thanks to its very high thermal conductivity is able to conduct away heat a thousand times better than any metallic conductor like copper.

In He II, the heat from a hot surface is carried away by the superfluid component, so in any design with complicated geometry and helium flows, the entire heat load acts on the phase interface. The boiling mechanism involves evaporation from surfaces ad in a flow of ordinary boiling liquid, the heat influx is uniformly distributed in unit volume of the two-phase mixture. In stratified He II, the heat influx is associated with the interface between the phases, so the He II evaporation rate is increased by a substantial factor. A major feature of boiling in He II is that the evaporation of the superfluid component predominates. The heat load is transported by convection in the superfluid component, and this consequently evaporates more rapidly than does the normal component.
When a two-phase flow of He II moves in a heated channel, a droplet structure or mist is formed in the vapor space as the amount of liquid in the stratified flow decreases. In a stratified flow of an ordinary liquid in a large- diameter tube, an increase in the bulk vapor content leads to the vapor becoming superheated and the liquid evaporating completely. In He II one prevents the vapor becoming superheated by encouraging the spontaneous formation of a droplet structure with a large heat-transfer surface, which provides a constant temperature over the channel cross section.
An efficient and quiet configuration for cooling the mirror by He II is the bain de Claudet. Here the idea is to provide superfluid helium at atmospheric pressure and to insure continuous refilling from the container of the helium in the normal state. In this way the He II bath is kept in a quiet hydrodynamic status well far from the boiling point . In this case the cold box on top of the seismic attenuator , ancillary to the main one  to which a the mirror is suspended, is an heat exchanger filled by superfluid helium at atmospheric pressure and operating in the stationary condition of almost zero mass flow.







