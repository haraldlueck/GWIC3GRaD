\section{Cryogenics}
\label{s:Cryogenics}
A feature of most designs for future interferometers, is the 
cryogenic~\footnote{which we define here to be less than 123\,K; the range from 80\,--\,123\,K may be considered as "high temperature cryogenics"} operation of the test mass mirrors\cite{ET, Voyager}.

In the GW field, there is a long history of using cryogenic resonant mass detectors\cite{coldBars}, as well as the recent operation of the CLIO and KAGRA interferometers in Japan.

\paragraph{Thermal Noise}\mbox{}\\
The primary benefit of low temperature operation of the mirrors is through the reduced thermal noise:
\begin{itemize}
\item Brownian noise of mirror substrate
\item Brownian noise of mirror coating
\item thermo-optic (thermo-elastic and thermo-refractive) noise
\end{itemize}
The relation between the dissipation and the power spectrum of the noise is described by Callen's Fluctuation-Dissipation Theorem~\cite{CaWe1951, Kubo:FDT, Callen:1959}:
\begin{equation}
S_x(f) = \frac{k_B T}{\pi^2 f^2} \left| Re \big[ Y(f) \big]\right|
\label{eq:FDT}
\end{equation}
and so we see that the displacement noise, $x(f)$, scales as $\sqrt{T}$. More significantly, many of the material properties of mirror substrates (e.g. sapphire and silicon) and coatings (e.g. GaAs, GaP, $\alpha$-Si), scale in a favorable way with decreasing temperature. This makes the improvement in the noise substantially larger than the naive $\sqrt{T}$ scaling.

\paragraph{Robust Operation}\mbox{}\\
In addition to the noise improvement, there are a number of operational issues affected by a low temperature environment:
\begin{enumerate}
\item Increased thermal conductivity in crystalline substrates; this dramatically reduced the thermal gradients in the mirror, and thereby, the induced wavefront distortions due to thermo-elastic deformations of the mirror surface and thermo-refractive lensing in the substrate bulk.

\item zero thermal expansion in silicon at 18 and 123\,K

\item ???
\end{enumerate}



\subsection{Current State of the Art}
\begin{itemize}
\item cryogenic bars
\item CLIO
\item KAGRA
\item silicon Fabry-Perot cavities for atomic clocks (JILA, PTB)
\end{itemize}
\subsection{Requirements}
\subsubsection{3G initial}
\subsubsection{future}
\subsection{Pathways and required facilities}
\subsection{Type of collaboration required:  small/large}
\subsection{Suggested mechanisms}
\subsection{Impact/relation to 2G and upgrades}